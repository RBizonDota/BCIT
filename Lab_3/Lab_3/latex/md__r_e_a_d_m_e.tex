\subsection*{Лабораторная работа по дисциплине БКИТ №X}

Выполнил\+: \href{https://vk.com/snipghost}{\tt Михаил Кучеренко \char`\"{}\+Snip\+Ghost\char`\"{}, ИУ5-\/34}, 18.\+10.\+2017 

~\newline


\subsubsection*{I. Описание задания}

Разработать программу, реализующую работу с классами.
\begin{DoxyEnumerate}
\item Программа должна быть разработана в виде консольного приложения на языке C\#.
\item Абстрактный класс «Геометрическая фигура» содержит виртуальный метод для вычисления площади фигуры.
\item Класс «Прямоугольник» наследуется от «Геометрическая фигура». Ширина и высота объявляются как свойства (property). Класс должен содержать конструктор по параметрам «ширина» и «высота».
\item Класс «Квадрат» наследуется от «Прямоугольник». Класс должен содержать конструктор по длине стороны.
\item Класс «Круг» наследуется от «Геометрическая фигура». Радиус объявляется как свойство (property). Класс должен содержать конструктор по параметру «радиус».
\item Для классов «Прямоугольник», «Квадрат», «Круг» переопределить виртуальный метод Object.\+To\+String(), который возвращает в виде строки основные параметры фигуры и ее площадь.
\item Разработать интерфейс I\+Print. Интерфейс содержит метод Print(), который не принимает параметров и возвращает void. Для классов «Прямоугольник», «Квадрат», «Круг» реализовать наследование от интерфейса I\+Print. Переопределяемый метод Print() выводит на консоль информацию, возвращаемую переопределенным методом To\+String(). 
\end{DoxyEnumerate}

~\newline


\subsubsection*{II. Код программы}


\begin{DoxyCode}
using System;
\textcolor{keyword}{using} System.Collections;
\textcolor{keyword}{using} System.Linq;
\textcolor{keyword}{using} System.Text;
\textcolor{keyword}{using} System.Threading.Tasks;

\textcolor{keyword}{namespace }\hyperlink{namespace_lab__3}{Lab\_3}
\{
    \textcolor{keyword}{class }Program
    \{

        \textcolor{keyword}{static} \textcolor{keywordtype}{void} Main()
        \{
            Menu m =\textcolor{keyword}{new} Menu();
            m.run();
        \}
    \}
\}
using System;

\textcolor{keyword}{namespace }\hyperlink{namespace_lab__3}{Lab\_3}
\{
    \textcolor{keyword}{interface }IPrint
    \{
        \textcolor{keywordtype}{void} print();
    \}
    \textcolor{keyword}{abstract} \textcolor{keyword}{class }Figure : IComparable, IPrint
    \{
        \textcolor{keyword}{virtual} \textcolor{keyword}{public} \textcolor{keywordtype}{double} square() \{ \textcolor{keywordflow}{return} 0; \}
        \textcolor{keyword}{public} \textcolor{keywordtype}{int} CompareTo(\textcolor{keywordtype}{object} obj)
        \{
            Figure f1 = obj as Figure;
            \textcolor{keywordflow}{if} (f1 == null)
                \textcolor{keywordflow}{throw} \textcolor{keyword}{new} ArgumentException(\textcolor{stringliteral}{"obj is not Figure!!!"});
            \textcolor{comment}{//if (this.square() > f1.square()) return 1;}
            \textcolor{comment}{//else if (this.square() == f1.square()) return 0;}
            \textcolor{comment}{//else return -1;}
            \textcolor{keywordflow}{return} this.square().CompareTo(f1.square());
        \}
        \textcolor{keyword}{abstract} \textcolor{keyword}{public} \textcolor{keywordtype}{void} print();

    \}
    \textcolor{keyword}{class }Circle : Figure
    \{
        \textcolor{keyword}{private} \textcolor{keywordtype}{double} r;
        \textcolor{keyword}{public} \textcolor{keywordtype}{double} radius
        \{
            \textcolor{keyword}{get}
            \{
                \textcolor{keywordflow}{return} r;
            \}
            \textcolor{keyword}{set}
            \{
                \textcolor{keywordflow}{if} (value < 0)
                    \textcolor{keywordflow}{throw} \textcolor{keyword}{new} ArgumentOutOfRangeException($\textcolor{stringliteral}{"\{nameof(value)\} должно быть положительным!"});
                r = value;
            \}
        \}
        \textcolor{keyword}{public} Circle(\textcolor{keywordtype}{double} r1)
        \{
            radius = r1;
        \}
        \textcolor{keyword}{public} \textcolor{keyword}{override} \textcolor{keywordtype}{string} ToString()
        \{
            \textcolor{keywordtype}{string} res;
            res = \textcolor{stringliteral}{"Class: Circle,"} +
               $\textcolor{stringliteral}{" Area: \{Math.Round(square(), 3)\}"};
            \textcolor{keywordflow}{return} res;
        \}
        \textcolor{keyword}{public} \textcolor{keyword}{override} \textcolor{keywordtype}{double} square()
        \{
            \textcolor{keywordflow}{return} Math.PI * r * r;
        \}
        \textcolor{keyword}{public} \textcolor{keyword}{override} \textcolor{keywordtype}{void} print()
        \{
            Console.WriteLine(this.ToString());
        \}

    \}
    \textcolor{keyword}{class }Rect : Figure
    \{
        \textcolor{keyword}{private} \textcolor{keywordtype}{double} a;
        \textcolor{keyword}{private} \textcolor{keywordtype}{double} b;
        \textcolor{keyword}{virtual} \textcolor{keyword}{public} \textcolor{keywordtype}{double} A
        \{
            \textcolor{keyword}{get}
            \{
                \textcolor{keywordflow}{return} a;
            \}
            \textcolor{keyword}{set}
            \{
                \textcolor{keywordflow}{if} (value < 0)
                    \textcolor{keywordflow}{throw} \textcolor{keyword}{new} ArgumentOutOfRangeException($\textcolor{stringliteral}{"\{nameof(value)\} должно быть положительным!"});
                a = value;
            \}
        \}
        \textcolor{keyword}{virtual} \textcolor{keyword}{public} \textcolor{keywordtype}{double} B
        \{
            \textcolor{keyword}{get}
            \{
                \textcolor{keywordflow}{return} b;
            \}
            \textcolor{keyword}{set}
            \{
                \textcolor{keywordflow}{if} (value < 0)
                    \textcolor{keywordflow}{throw} \textcolor{keyword}{new} ArgumentOutOfRangeException($\textcolor{stringliteral}{"\{nameof(value)\} должно быть положительным!"});
                b = value;
            \}
        \}
        \textcolor{keyword}{public} Rect(\textcolor{keywordtype}{double} a1, \textcolor{keywordtype}{double} b1)
        \{
            A = a1;
            B = b1;
        \}
        \textcolor{keyword}{public} \textcolor{keyword}{override} \textcolor{keywordtype}{string} ToString()
        \{
            \textcolor{keywordtype}{string} res;
            res = \textcolor{stringliteral}{"Class: Rectangle,"} +
               $\textcolor{stringliteral}{" Area: \{Math.Round(square(),3)\}"};
            \textcolor{keywordflow}{return} res;
        \}
        \textcolor{keyword}{public} \textcolor{keyword}{override} \textcolor{keywordtype}{double} square()
        \{
            \textcolor{keywordflow}{return} A * B;
        \}
        \textcolor{keyword}{public} \textcolor{keyword}{override} \textcolor{keywordtype}{void} print()
        \{
            Console.WriteLine(this.ToString());
        \}
    \}
    \textcolor{keyword}{class }Square : Rect
    \{
        \textcolor{keyword}{private} \textcolor{keywordtype}{double} a;
        \textcolor{comment}{//public override double A}
        \textcolor{comment}{//\{}
        \textcolor{comment}{//    get}
        \textcolor{comment}{//    \{}
        \textcolor{comment}{//        return a;}
        \textcolor{comment}{//    \}}
        \textcolor{comment}{//    set}
        \textcolor{comment}{//    \{}
        \textcolor{comment}{//        if (value < 0)}
        \textcolor{comment}{//            throw new ArgumentOutOfRangeException($"\{nameof(value)\} должно быть положительным!");}
        \textcolor{comment}{//        a = value;}
        \textcolor{comment}{//    \}}
        \textcolor{comment}{//\}}
        \textcolor{keyword}{public} \textcolor{keyword}{override} \textcolor{keywordtype}{string} ToString()
        \{
            \textcolor{keywordtype}{string} res;
            res = \textcolor{stringliteral}{"Class: Square,"} +
               $\textcolor{stringliteral}{" Area: \{Math.Round(square(), 3)\}"};
            \textcolor{keywordflow}{return} res;
        \}
        \textcolor{comment}{//public override double square()}
        \textcolor{comment}{//\{}
        \textcolor{comment}{//    return A * A;}
        \textcolor{comment}{//\}}
        \textcolor{keyword}{public} Square(\textcolor{keywordtype}{double} a1) : base(a1,a1)
        \{
        \}
    \}
\}
using System;
\textcolor{keyword}{using} System.Collections;
\textcolor{keyword}{using} System.Collections.Generic;
\textcolor{keyword}{using} System.Linq;
\textcolor{keyword}{using} System.Text;
\textcolor{keyword}{using} System.Threading.Tasks;

\textcolor{keyword}{namespace }\hyperlink{namespace_lab__3}{Lab\_3}
\{

    \textcolor{keyword}{class }TestArray : IPrint
    \{
        ArrayList arr1;
        Random rand;
        \textcolor{keyword}{public} TestArray()
        \{
            arr1 = \textcolor{keyword}{new} ArrayList();
            rand = \textcolor{keyword}{new} Random();
        \}
        \textcolor{keyword}{public} \textcolor{keywordtype}{void} Gener(\textcolor{keywordtype}{int} rmax, \textcolor{keywordtype}{int} times)
        \{
            \textcolor{keywordflow}{for} (\textcolor{keywordtype}{int} i = 0; i < times; i++)
            \{
                Figure res = null;
                \textcolor{keywordtype}{int} a = rand.Next(3);
                \textcolor{keywordflow}{switch} (a)
                \{
                    \textcolor{keywordflow}{case} 0:
                        res = \textcolor{keyword}{new} Circle(rand.Next(rmax));
                        \textcolor{keywordflow}{break};
                    \textcolor{keywordflow}{case} 1:
                        res = \textcolor{keyword}{new} Rect(rand.Next(rmax), rand.Next(rmax));
                        \textcolor{keywordflow}{break};
                    \textcolor{keywordflow}{case} 2:
                        res = \textcolor{keyword}{new} Square(rand.Next(rmax));
                        \textcolor{keywordflow}{break};
                \}
                arr1.Add(res);
            \}
        \}
        \textcolor{keyword}{public} \textcolor{keywordtype}{void} print()
        \{
            \textcolor{keywordflow}{foreach} (Figure f \textcolor{keywordflow}{in} arr1)
            \{
                f.print();
            \}
        \}
        \textcolor{keyword}{public} \textcolor{keywordtype}{void} Sort()
        \{
            arr1.Sort();
        \}
    \}
    \textcolor{comment}{//==================================}
    \textcolor{keyword}{public} \textcolor{keyword}{interface }IMatrAdapter<T>
    \{
       T getEmptyElement();
       \textcolor{keywordtype}{bool} checkEmptyElement(T element);
    \}
    \textcolor{keyword}{public} \textcolor{keyword}{class }FigureMatrixCheckEmpty : IMatrAdapter<Figure>
    \{
        Figure IMatrAdapter<Figure>.getEmptyElement()
        \{
            \textcolor{keywordflow}{return} null;
        \}
        \textcolor{keywordtype}{bool} IMatrAdapter<Figure>.checkEmptyElement(Figure element)
        \{
            \textcolor{keywordtype}{bool} res = \textcolor{keyword}{false};
            \textcolor{keywordflow}{if} (element == null)
                res = \textcolor{keyword}{true};
            \textcolor{keywordflow}{return} res;
        \}
    \}
    \textcolor{keyword}{class }Matrix<T>:IPrint
    \{
        \textcolor{keywordtype}{int} xmax, ymax, zmax;
        \textcolor{keyword}{public} IMatrAdapter<T> Adap;
        Dictionary<string, T> \_matr = \textcolor{keyword}{new} Dictionary<string, T>();
        \textcolor{keyword}{public} Matrix(\textcolor{keywordtype}{int} \_xmax, \textcolor{keywordtype}{int} \_ymax, \textcolor{keywordtype}{int} \_zmax, IMatrAdapter<T> IMCE)
        \{
            xmax = \_xmax;
            ymax = \_ymax;
            zmax = \_zmax;
            Adap = IMCE;
        \}
        \textcolor{keyword}{public} T \textcolor{keyword}{this}[\textcolor{keywordtype}{int} x, \textcolor{keywordtype}{int} y, \textcolor{keywordtype}{int} z]
        \{
            \textcolor{keyword}{set}
            \{
                CheckBounds(x, y, z);
                \textcolor{keywordtype}{string} key = DictKey(x, y, z);
                \textcolor{keywordflow}{if} (this.\_matr.ContainsKey(key))
                \{
                    this.\_matr[key] = value;
                \}
                \textcolor{keywordflow}{else}
                \{
                    this.\_matr.Add(key, value);
                \}
            \}
            \textcolor{keyword}{get}
            \{
                CheckBounds(x, y, z);
                \textcolor{keywordtype}{string} key = DictKey(x, y, z);
                \textcolor{keywordflow}{if} (this.\_matr.ContainsKey(key))
                \{
                    \textcolor{keywordflow}{return} this.\_matr[key];
                \}
                \textcolor{keywordflow}{else}
                \{
                    \textcolor{keywordflow}{return} this.Adap.getEmptyElement();
                \}
            \}
        \}
        \textcolor{keywordtype}{void} CheckBounds(\textcolor{keywordtype}{int} x, \textcolor{keywordtype}{int} y, \textcolor{keywordtype}{int} z)
        \{
            \textcolor{keywordflow}{if} (x < 0 || x >= xmax)
                \textcolor{keywordflow}{throw} \textcolor{keyword}{new} ArgumentOutOfRangeException(\textcolor{stringliteral}{"WRONG x"});
            \textcolor{keywordflow}{if} (y < 0 || y >= ymax)
                \textcolor{keywordflow}{throw} \textcolor{keyword}{new} ArgumentOutOfRangeException(\textcolor{stringliteral}{"WRONG y"});
            \textcolor{keywordflow}{if} (z < 0 || z >= zmax)
                \textcolor{keywordflow}{throw} \textcolor{keyword}{new} ArgumentOutOfRangeException(\textcolor{stringliteral}{"WRONG z"});

        \}
        \textcolor{keywordtype}{string} DictKey(\textcolor{keywordtype}{int} x, \textcolor{keywordtype}{int} y, \textcolor{keywordtype}{int} z)
        \{
            \textcolor{keywordflow}{return} x.ToString() + \textcolor{stringliteral}{"\_"} + y.ToString()+\textcolor{stringliteral}{"\_"}+ z.ToString();
        \}
        \textcolor{keyword}{public} \textcolor{keyword}{override} \textcolor{keywordtype}{string} ToString()
        \{
            StringBuilder res = \textcolor{keyword}{new} StringBuilder();
            res.Append(\textcolor{stringliteral}{"x | y | z | value\(\backslash\)n"});
            \textcolor{keywordflow}{for} (\textcolor{keywordtype}{int} i = 0; i < xmax; i++)
            \{
                \textcolor{keywordflow}{for} (\textcolor{keywordtype}{int} j = 0; j < ymax; j++)
                \{
                    \textcolor{keywordflow}{for} (\textcolor{keywordtype}{int} k = 0; k < zmax; k++)
                    \{
                        \textcolor{keywordflow}{if} (!Adap.checkEmptyElement(\textcolor{keyword}{this}[i, j, k]))
                        \{
                            res.Append(i.ToString()+\textcolor{stringliteral}{" | "}+j.ToString() + \textcolor{stringliteral}{" | "}+k.ToString() + \textcolor{stringliteral}{" | "}+\textcolor{keyword}{this}[i,
       j, k].ToString()+\textcolor{stringliteral}{"\(\backslash\)n"});
                        \}
                    \}
                \}
            \}
            res.Append(\textcolor{stringliteral}{"All other elements are empty\(\backslash\)n"});
            \textcolor{keywordflow}{return} res.ToString();
        \}
        \textcolor{keyword}{public} \textcolor{keywordtype}{void} print()
        \{
            Console.WriteLine(this.ToString());
        \}
    \}
    \textcolor{comment}{//==================================}
    \textcolor{keyword}{class }SimpleList<T> : IPrint
        where T : IComparable
    \{
        \textcolor{keyword}{protected} SimpleListItem<T> first;
        \textcolor{keyword}{protected} SimpleListItem<T> last;
        \textcolor{keywordtype}{int} \_count;
        \textcolor{keyword}{public} SimpleList()
        \{
            first = null;
            last = null;
        \}
        \textcolor{keyword}{public} \textcolor{keywordtype}{int} count
        \{
            \textcolor{keyword}{get}
            \{
                \textcolor{keywordflow}{return} \_count;
            \}
            \textcolor{keyword}{protected} \textcolor{keyword}{set}
            \{
                \_count = value;
            \}
        \}
        \textcolor{keyword}{public} \textcolor{keywordtype}{void} Add(T element)
        \{
            SimpleListItem<T> a = \textcolor{keyword}{new} SimpleListItem<T>(element);
            \textcolor{keywordflow}{if} (this.last == null)
            \{
                first = a;
                last = a;
            \}
            \textcolor{keywordflow}{else}
            \{
                last.next = a;
                last = a;
            \}
            count++;
        \}
        \textcolor{keyword}{public} SimpleListItem<T> getItem(\textcolor{keywordtype}{int} number)
        \{
            \textcolor{keywordflow}{if} (number < 0 || number >= count)
            \{
                \textcolor{keywordflow}{throw} \textcolor{keyword}{new} ArgumentException(\textcolor{stringliteral}{"WRONG INDEX"});
            \}
            \textcolor{keywordflow}{else}
            \{
                \textcolor{keywordtype}{int} i = 0;
                SimpleListItem<T> buf = this.first;
                \textcolor{keywordflow}{while} (i<number)
                \{
                    i++;
                    buf = buf.next;
                \}
                \textcolor{keywordflow}{return} buf;
            \}
        \}
        \textcolor{keyword}{public} T Get(\textcolor{keywordtype}{int} number)
        \{
            \textcolor{keywordflow}{return} getItem(number).data;
        \}
        \textcolor{keyword}{public} \textcolor{keyword}{override} \textcolor{keywordtype}{string} ToString()
        \{
            StringBuilder res = \textcolor{keyword}{new} StringBuilder();
            \textcolor{keywordflow}{for} (\textcolor{keywordtype}{int} i = 0; i < count; i++)
            \{
                res.Append(i.ToString()+\textcolor{stringliteral}{") "}+Get(i).ToString()+\textcolor{stringliteral}{"\(\backslash\)n"});
            \}
            \textcolor{keywordflow}{return} res.ToString();
        \}
        \textcolor{keyword}{public} \textcolor{keywordtype}{void} print()
        \{
            Console.WriteLine(this.ToString());
        \}
    \}
    \textcolor{keyword}{class }SimpleListItem<T>
        where T : IComparable
    \{
        \textcolor{keyword}{public} SimpleListItem(T param)
        \{
            data = param;
        \}
        \textcolor{keyword}{public} T data
        \{
            \textcolor{keyword}{get}
            \{
                \textcolor{keywordflow}{return} \_data;
            \}
            \textcolor{keyword}{set}
            \{
                \_data = value;
            \}
        \}
        T \_data;
        \textcolor{keyword}{public} SimpleListItem<T> next
        \{
            \textcolor{keyword}{get}
            \{
                \textcolor{keywordflow}{return} \_next;
            \}
            \textcolor{keyword}{set}
            \{
                \_next = value;
            \}
        \}
        SimpleListItem<T> \_next = null;
    \}
    \textcolor{keyword}{class }SimpleStack<T> : SimpleList<T>
        where T:IComparable
    \{
        \textcolor{keyword}{public} SimpleListItem<T> Pop(\textcolor{keywordtype}{int} i)
        \{
            SimpleListItem<T> buf = getItem(i);
            \textcolor{keywordflow}{if} (first == buf)
            \{
                first = buf.next;\textcolor{comment}{//ошибка, если в списке 1 элемент!}
                \textcolor{keywordflow}{if} (count == 1)
                    last = null;
            \}
            \textcolor{keywordflow}{else}
            \{
                getItem(i - 1).next = buf.next;
            \}
            count--;
            \textcolor{keywordflow}{return} buf;
        \}
        \textcolor{keyword}{public} \textcolor{keywordtype}{void} Push(T element)
        \{
            Add(element);
        \}
    \}
    \textcolor{comment}{//==================================}
    \textcolor{keyword}{class }Menu
    \{
        MySystem mySys = \textcolor{keyword}{new} MySystem();
        \textcolor{keywordtype}{int} intReadKey()
        \{
            \textcolor{keywordtype}{int} res;
            \textcolor{keywordtype}{bool} f = \textcolor{keywordtype}{int}.TryParse(Console.ReadLine(), out res);
            \textcolor{keywordflow}{if} (!f)
                \textcolor{keywordflow}{throw} \textcolor{keyword}{new} ArgumentException(\textcolor{stringliteral}{"Wrong input"});
            \textcolor{keywordflow}{return} res;
        \}
        \textcolor{keywordtype}{void} case1()
        \{
            Console.Clear();
            Console.Write(\textcolor{stringliteral}{"Enter number of elements. Array will be filled by randomly generated figures: "})
      ;
            TestArray a = \textcolor{keyword}{new} TestArray();
            a.Gener(10, intReadKey());
            mySys.printPalka(20);
            a.print();
            mySys.printPalka(20);
            Console.WriteLine(\textcolor{stringliteral}{"Sorting..."});
            mySys.printPalka(20);
            a.Sort();
            a.print();
        \}
        \textcolor{keywordtype}{void} case2()
        \{
            Console.Clear();
            Console.Write(\textcolor{stringliteral}{"Enter maximum size: "});
            \textcolor{keywordtype}{int} rmax= intReadKey();
            Console.Write(\textcolor{stringliteral}{"Enter number of elements generated: "});
            \textcolor{keywordtype}{int} times = intReadKey();
            mySys.printPalka(20);
            Random rand = \textcolor{keyword}{new} Random();
            Matrix<Figure> matr = \textcolor{keyword}{new} Matrix<Figure>(rmax,rmax,rmax,\textcolor{keyword}{new} FigureMatrixCheckEmpty());
            \textcolor{keywordflow}{for} (\textcolor{keywordtype}{int} i = 0; i < times; i++)
            \{
                \textcolor{keywordtype}{int} i0 = rand.Next(rmax), i1 = rand.Next(rmax), i2 = rand.Next(rmax);
                matr[i0, i1, i2] = mySys.GenerFigure(10, rand);
            \}
            Console.WriteLine(\textcolor{stringliteral}{"Printing existing elements..."});
            mySys.printPalka(20);
            matr.print();
        \}
        \textcolor{keywordtype}{void} case3()
        \{
            Console.Clear();
            Console.Write(\textcolor{stringliteral}{"Enter number of elements generated: "});
            \textcolor{keywordtype}{int} times = intReadKey();
            SimpleStack<Figure> a = \textcolor{keyword}{new} SimpleStack<Figure>();
            Random rand = \textcolor{keyword}{new} Random();
            \textcolor{keywordflow}{for} (\textcolor{keywordtype}{int} i = 0; i < times; i++)
            \{
                a.Push(mySys.GenerFigure(10, rand));
            \}
            Console.WriteLine(\textcolor{stringliteral}{"Printing generated data..."});
            mySys.printPalka(20);
            a.print();
            mySys.printPalka(20);
            Console.Write(\textcolor{stringliteral}{"Enter number of eliminating element: "});
            \textcolor{keywordtype}{bool} t = \textcolor{keyword}{true};
            SimpleListItem<Figure> f = null;
            \textcolor{keywordflow}{while} (t)
            \{
                times = intReadKey();
                \textcolor{keywordflow}{try}
                \{
                    f = a.Pop(times);
                    t = \textcolor{keyword}{false};
                \}
                \textcolor{keywordflow}{catch} (ArgumentException)
                \{
                    Console.Write(\textcolor{stringliteral}{"Wrong input! Please, enter again: "});
                    \textcolor{keywordflow}{continue};
                \}
            \}
            mySys.printPalka(20);
            Console.WriteLine(\textcolor{stringliteral}{"Eliminated element : "} + f.data.ToString());
            mySys.printPalka(20);
            Console.WriteLine(\textcolor{stringliteral}{"Printing stack: "});
            a.print();
        \}
        \textcolor{keyword}{public} \textcolor{keywordtype}{void} run()
        \{
            \textcolor{keywordtype}{bool} t = \textcolor{keyword}{true};
            \textcolor{keywordtype}{bool} p = \textcolor{keyword}{true};
            \textcolor{keywordflow}{while} (p)
            \{
                Console.Clear();
                \textcolor{keywordflow}{if} (t) Console.WriteLine(\textcolor{stringliteral}{"Lab\_3 by Baybarin R.\(\backslash\)n"});
                t = \textcolor{keyword}{false};
                Console.Write(\textcolor{stringliteral}{"============\(\backslash\)n"} +
                              \textcolor{stringliteral}{"Menu\(\backslash\)n"} +
                              \textcolor{stringliteral}{"============\(\backslash\)n"} +
                              \textcolor{stringliteral}{"1) Testing ArrayList.Sort()\(\backslash\)n"} +
                              \textcolor{stringliteral}{"2) Testing Sparse Matrix\(\backslash\)n"} +
                              \textcolor{stringliteral}{"3) Testing SimpleList\(\backslash\)n"} +
                              \textcolor{stringliteral}{"0) Exit\(\backslash\)n"} +
                              \textcolor{stringliteral}{"============\(\backslash\)n"} +
                              \textcolor{stringliteral}{"Your choice: "});
                \textcolor{keywordtype}{int} caseSwitch;
                \textcolor{keywordtype}{bool} f;
                f = \textcolor{keywordtype}{int}.TryParse(Console.ReadLine(), out caseSwitch);
                \textcolor{keywordflow}{if} (!f || caseSwitch < 0 || caseSwitch > 3)
                \{
                    Console.WriteLine(\textcolor{stringliteral}{"ERROR!!!!! WRONG INPUT"});
                    mySys.pause();
                    \textcolor{keywordflow}{continue};
                \}
                \textcolor{keywordflow}{switch} (caseSwitch)
                \{
                    \textcolor{keywordflow}{case} 0:
                        p = \textcolor{keyword}{false};
                        Console.WriteLine(\textcolor{stringliteral}{"Exiting...Press Enter"});
                        \textcolor{keywordflow}{break};
                    \textcolor{keywordflow}{case} 1: case1();
                        \textcolor{keywordflow}{break};
                    \textcolor{keywordflow}{case} 2: case2();
                        \textcolor{keywordflow}{break};
                    \textcolor{keywordflow}{case} 3: case3();
                        \textcolor{keywordflow}{break};
                \}
                mySys.pause();
            \}
        \}
    \}
\}
using System;
\textcolor{keyword}{using} System.Collections.Generic;
\textcolor{keyword}{using} System.Linq;
\textcolor{keyword}{using} System.Text;
\textcolor{keyword}{using} System.Threading.Tasks;

\textcolor{keyword}{namespace }\hyperlink{namespace_lab__3}{Lab\_3}
\{
    \textcolor{keyword}{class }MySystem
    \{
        \textcolor{keyword}{public} \textcolor{keywordtype}{void} pause()
        \{
            Console.WriteLine(\textcolor{stringliteral}{"Input enter..."});
            Console.ReadLine();
        \}
        \textcolor{keyword}{public} \textcolor{keywordtype}{void} printPalka(\textcolor{keywordtype}{int} l)
        \{
            \textcolor{keywordflow}{for} (\textcolor{keywordtype}{int} i = 0; i < l; i++)
            \{
                Console.Write(\textcolor{stringliteral}{"="});
            \}
            Console.Write(\textcolor{stringliteral}{"\(\backslash\)n"});
        \}
        \textcolor{keyword}{public} Figure GenerFigure(\textcolor{keywordtype}{int} param, Random rand)
        \{
            Figure res = null;
            \textcolor{keywordtype}{int} a = rand.Next(3);
            \textcolor{keywordflow}{switch} (a)
            \{
                \textcolor{keywordflow}{case} 0:
                    res = \textcolor{keyword}{new} Circle(rand.Next(param));
                    \textcolor{keywordflow}{break};
                \textcolor{keywordflow}{case} 1:
                    res = \textcolor{keyword}{new} Rect(rand.Next(param), rand.Next(param));
                    \textcolor{keywordflow}{break};
                \textcolor{keywordflow}{case} 2:
                    res = \textcolor{keyword}{new} Square(rand.Next(param));
                    \textcolor{keywordflow}{break};
            \}
            \textcolor{keywordflow}{return} res;
        \}
    \}
\}
\end{DoxyCode}


\href{Program.cs}{\tt Исходный код} 

~\newline


\subsubsection*{I\+II. Примеры работы}

~\newline
